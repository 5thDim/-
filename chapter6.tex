\documentclass[main.tex]{subfiles}
\begin{document}
\begin{center}
    {\LARGE \bf 第六章\,\,当代日常生活理论}\\
    Fall 2025
\end{center}

\section{符号互动论}
\paragraph{基本假设和原则}
\begin{itemize}
    \item 人们对事物做出反应,依据是事物的意义。
    \item 意义源于人们之间的互动。
    \item 人们通过社会互动可以持续不断地调整意义。
    \item 人可以和自己互动(?);
    \item 人类在情境中行动是有意义的。人们赋予情境以意义,再根据意义互动。
    \item 社会由参与社会互动的人组成。
\end{itemize}

符号互动论看起来是一种极其微观的理论,它关注人们在具体情境中的反应。这个学派的工作者罗伯特·帕克非常强调使用观察的方法,做一些田野工作。

查尔斯·库利提出了一些相关概念, which 我并不知道为什么这本书要在这里讲。摘录如下:
\begin{itemize}
    \item 镜中自我。人们通过镜子形成对自我的认知。在社会化的过程中,他人就是镜子。
    \item 原初群体。指个体在社会化时发挥关键作用的亲密群体。镜中自我通常实在原初群体中形成的。
    \item 同情自省法。就是说做田野的时候要设身处地地为研究对象着想。
\end{itemize}

\section{拟剧论}
自我是符号互动论里的重要元素,自我可以成为行动的对象。论述自我最重要的理论是戈夫曼的拟剧论。戈夫曼收到米德的“主我与宾我之间的张力”的影响提出了拟剧论。他认为他人的期待和自发性行动的差别造成了这种张力。为了应付这种张力,人们开始在社会观众前表演。

拟剧论认为,社会场景就像戏剧,人们在戏剧中呈现自我,构建一些被演绎出来的形象,这叫做印象管理。一些关键概念是前台、场景、个人前台、外表、举止;后台、外部。

戈夫曼认为,由于人们像呈现一些理想情况,就不可避免的在表演中进行隐瞒、伪装。行动者偶尔也会想表现得贴近观众,和观众有特殊联系。行动者也会采取神秘化的方法。有时,表演者和观众会形成某种团队,共同维持这一种戏剧。

污名和声誉。戈夫曼也研究被污名化的群体如何应对这样的身份,与其他个体互动。

印象管理。戈夫曼也具体研究了很多印象管理的实例。

\section{常人方法学}
常人方法学研究的是在日常情境中的常人运用日常方法进行行为的过程。

以往的社会学家认为,行动者根据各种各样的social fact而行动,而几乎没有自己独立判断。加芬克尔强烈地反抗这样的观点,认为常人组织日常生活的方法是特别的,具有很大的研究价值。本书作者认为,这样的视角不太微观也不太宏观。

加芬克尔的一大重点是,研究“说明”这一动作本身。我们把说明理解为“人们解释特殊情境的方式”。我们可以研究“说明”这一动作如何被运用到实践行动中。同时,我们指出,社会学研究本身也在“说明”,所以同样也要接受审查。

常人方法学的一种研究方法是,研究人们如何应对微小的秩序破坏。在这样的应对过程中,人们往往会选择极其富有弹性的处理。人们希望迅速采取行动,将破坏正常化,以回到熟悉的情境中。

另一个案例是性别实现。加芬克尔想通过这个案例说明,社会身份是需要获得的。我们需要学习一系列的社会技能,才能在社会情境中担任某种身份。



\end{document}