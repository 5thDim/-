\documentclass[main.tex]{subfiles}
\begin{document}
\begin{center}
    {\LARGE \bf 第四章\,\,当代大理论}\\
    Fall 2025
\end{center}

\section{结构功能主义}
结构功能主义者关心社会结构之间的关系,主要是社会系统和系统之间的相互作用(提供的功能),尽管可能社会内各系统间可能会有冲突。

\subsection{社会分层的功能理论}
戴维斯和莫尔指出,社会分层是普遍而必要的。社会结构中的不同位置具有不同的声望,社会激励人们进入分层系统中恰当的位置。他们还认为,某些位置更加重要,并且占据者并不会那么愉快,应该给予更多的报酬以吸引人们(?);而较低的位置被假定有更多的舒适感,也对职位的要求较低。

我并不理解为什么会有这样的见解……也许是有一些历史背景在的吧……

所以社会分层就是一种位置等级结构,它会分配不同的系统以不同的社会成员,以完成各种各样的功能。

对这个理论有很多批判。一个常见的见解是,分层的功能理论可能造成阶级固化;分层这一预言本身也值得质疑,未来社会可能是不分层的;还有关于社会位置的价值取向更是很不让人理解。

\subsection{帕森斯}
这里讨论的是帕森斯晚年的结构功能主义。

帕森斯认为,每个行动系统都应该具有这样的四个功能,分别是适应、目标达成、整合和模式维持,即AGIL。详细地说,系统应当适应环境、指定一些需要达成的目标、调节系统内各部分、维持个体积极性。分别都是好理解的功能,但特别令人注意的是第四项能和前三项并列……模式维持潜藏在社会分层功能理论里,但在这里以更宏观的形式出现。

帕森斯讨论了一些general的行动系统,并给予了他们对应承担的功能,包括行为有机体、人格系统、社会系统和文化系统。每种行动系统都可以有不同大小的结构。比方说行为有机体,最小结构的行为有机体是人的身体,大一点的也许是人群、社会组织,最终到终极现实(形而上学……)。

基于这样的理论结构,帕森斯提出了一些假设,以回答秩序问题。这里不写了。

帕森斯关于社会系统的定义是:在某种环境下互动的、一定数量的人类行动者。他的重点在于互动系统,使用“角色-地位”作为社会的基本单元。看起来,帕森斯关心社会系统里的一个个位置,以及这个位置的个体身份,但他并不真的关心每一个人。帕森斯也关心系统如何整合系统内的行动者。不过帕森斯还是给予了个体相对的自由。比如说,帕森斯不认为不太强的个体差异会影响社会化过程和社会秩序,因为社会具有一定的弹性,同时也应该具有一定的开放性和多元性。

就这样,在结构化和一定的多元化控制之下,社会秩序得已建立起来。

接着帕森斯也讨论了其他行动系统。文化系统,一方面它是一个相对独立的行动系统,另一方面它也可以承担起模式维持的功能。人格系统,他是一个相对消极的形象,会在社会系统中根据各种各样的刺激做出反应。行为有机体,不太被帕森斯关心。

总之,帕森斯的分析框架是比较权威的。

\subsection{罗伯特·默顿}
默顿关心社会中层结构,他的政治光谱有点偏左翼。

默顿的想法是很容易理解的。默顿认为,往常的功能分析理论里有一些假定具有较差的现实性,比方说,认为现代社会的结构都具有一定的正面功能、都接受了高度整合、都不可或缺且已经达成最好的状况。这些假定都明显不太现实。默顿提出了一些观点以改变这些假设。

默顿看上去有一些经验主义和实证主义。他认为,分析的对象应该是“标准化的”,这是一种社会功能主义。默顿区分了功能、功能失调和非功能。为了权衡功能和功能失调,默顿提出净平衡概念,功能分析层次概念。默顿又区分了显性功能、隐形功能,预期后果和非预期后果。

举个例子。按照这些概念,一个结构可能对系统整体是功能失调的,但他可能是对系统内某个部分具有功能性,所以才能存在,但这样的结构不是不可或缺的,这提供了社会改良的理论依据。

\section{冲突理论}
冲突理论在一定程度上可以看成是对结构功能主义的回应(看起来像是所有的假设反着来,但意外地得到了相同地结构……?)所以,这本书地作者评价说,它更像是改头换面地结构功能主义,而不是社会批判理论。后期,它逐渐被西方新马克思主义所替代。

\subsection{达伦多夫}
冲突论者认为,社会的每一处都处于变动和冲突之中,无处没有利益争夺,但社会顶层的成员强制执行了社会秩序。

作为代表人,达伦科夫认为,社会理论应该分成冲突论和共识论,两方面存在各自的靠量立场。在冲突论的观点下,社会靠强制性的约束形成秩序,而系统性的社会冲突大多室友权力的差异化分配造成的。

达伦科夫认为,不同社会位置掌握不同的权威,权威依赖于位置而不是位置上的人。应当研究各个位置组成的结构,以及位置之间的冲突。上位者基于群众的期待控制下属(?),并对不服从的群众施以惩罚。社会由大量强制性协作的社群组成,社群内部只有上位者和丛书这的利益发生冲突,权力的合法性始终不稳定。

达伦科夫区分了准群体、利益群体和冲突群体。

总之,达伦科夫认为,冲突群体一旦形成,他们就会参与行动,导致社会结构发生变化。


\end{document}